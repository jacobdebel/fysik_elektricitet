% Intended LaTeX compiler: xelatex
\documentclass[a4paper, 12pt]{article}
\usepackage{graphicx}
\usepackage{longtable}
\usepackage{wrapfig}
\usepackage{rotating}
\usepackage[normalem]{ulem}
\usepackage{amsmath}
\usepackage{amssymb}
\usepackage{capt-of}
\usepackage{hyperref}
\usepackage[danish]{babel}
\usepackage{mathtools}
\usepackage[margin=3.0cm]{geometry}
\hypersetup{colorlinks, linkcolor=black, urlcolor=blue}
\setlength{\parindent}{0em}
\parskip 1.5ex
\author{Jacob Debel}
\date{Fysik B}
\title{En modstands temperaturafhængighed\\\medskip
\large Elektricitet}
\hypersetup{
 pdfauthor={Jacob Debel},
 pdftitle={En modstands temperaturafhængighed},
 pdfkeywords={},
 pdfsubject={},
 pdfcreator={Emacs 29.4 (Org mode 9.6.15)}, 
 pdflang={Danish}}
\begin{document}

\maketitle

\section*{Motivation}
\label{sec:orgdf66835}
Der er en klar sammenhæng mellem en elektrisk strøm igennem en metallisk leder og varmeudvikling, hvilken er sammenfattet i Joules lov.

I 1900 antog den tyske fysiker P. Drude at ledningselektronerne støder ind i ionerne i lederen, hvilket bevirker, at ionerne begynder at vibrere, og lederen dermed bliver varm.
Ergo ved at sende større og større strømme gennem en leder vil temperaturen på lederen stige.

Spørgsmålet er, om det omvendte også er tilfældet. \textbf{Øges en leders modstand, hvis man øger dennes temperatur?}

\section*{Teori}
\label{sec:org8eda2f1}
En model for en modstands temperaturafhængighed lyder som følger:

$$R_t = R_0 \left( 1 + \alpha \left( t - t_0 \right) \right)\,,$$

hvor \(R_t\) er lederens modstand ved temperaturen \(t\) og \(R_0\) er modstanden ved temperaturen \(t_0\).

Det er denne sammenhæng I skal forsøge at eftervise.


\section*{Udstyr}
\label{sec:org656e7a9}
\begin{itemize}
\item Termometer
\item Ohmmeter
\item 3 forskellige ledere (prober)
\item ledninger
\item 2 store bægerglas
\item Varmeapparat
\end{itemize}

\section*{Fremgangsmåde}
\label{sec:org3624635}
\begin{itemize}
\item Et bægerglas et fyldt med vand, som er sat til at “simre” over varmeapparatet.
\item Det andet bæreglas er fyldt med koldt vand fra hanen.
\item En afkølet probe (opbevaret i det “kolde” bægerglas) forbindes til ohmmeteret og proben nedsænkes i det varme vand.
\item Noter løbende sammenhørende værdier af modstand og temperatur.
\item Udfør forsøget for de 3 forskellige prober.
\item Målingerne indsættes i et regneark og der tegnes en graf over modstanden som funktion af temperaturen.
\end{itemize}

Vi er nu klar til at bestemme temperaturkoefficienterne for de tre prober ved en senere undervisningslektion.
\end{document}
